\documentclass[12pt]{article}
\usepackage[spanish]{babel}
\usepackage{makeidx}
\usepackage[margin=1in]{geometry}  % set the margins to 1in on all sides
\usepackage{graphicx}              % to include figures
\usepackage{amsmath}               % great math stuff
\usepackage{amsfonts}              % for blackboard bold, etc
\usepackage{amsthm}                % better theorem environments
\usepackage{makeidx}               % index
\usepackage[utf8]{inputenc}        % now we have tildes!
\usepackage{wrapfig}               % images
\usepackage{listings}              % Unordered lists
\usepackage{hyperref}              % hyperlinks
\usepackage{xcolor}                % to colorize font
\usepackage{blindtext}             % to colorize font

\makeatletter
\renewcommand\thesection{}
\makeatother

\makeindex

\begin{document}

\begin{titlepage}

\newcommand{\HRule}{\rule{\linewidth}{0.5mm}} % Defines a new command for the horizontal lines, change thickness here

\center % Center everything on the page

%----------------------------------------------------------------------------------------
%	HEADING SECTIONS
%----------------------------------------------------------------------------------------

\textsc{\LARGE Universidad Carlos III de Madrid}\\[1.2cm] % Name of your university/college

\includegraphics[width=9cm]{Logo.png}\\[1.2cm] % Include a department/university logo - this will require the graphicx package

\textsc{\Large Aprendizaje Automático}\\[0.5cm] % Major heading such as course name
\textsc{\large Grado en Ingeniería Informática}\\[0.6cm] % Minor heading such as course title
\textsc{\large Grupo 83}\\[0.5cm]

%----------------------------------------------------------------------------------------
%	TITLE SECTION
%----------------------------------------------------------------------------------------

\HRule \\[0.7cm]
{ \huge \bfseries Tutorial 3: Experimentación Múltiple}\\[0.4cm] % Title of your document
\HRule \\[1.3cm]

%----------------------------------------------------------------------------------------
%	AUTHOR SECTION
%----------------------------------------------------------------------------------------


% If you don't want a supervisor, uncomment the two lines below and remove the section above
\emph{Autores:}\\
Daniel \textsc{Medina García}\\ % Your name
Alejandro \textsc{Rodríguez Salamanca}\\[1.5cm] % Your name

%----------------------------------------------------------------------------------------
%	DATE SECTION
%----------------------------------------------------------------------------------------

{\large \today}\\ % Date, change the \today to a set date if you want to be precise

\vfill % Fill the rest of the page with whitespace

\end{titlepage}

\tableofcontents

\newpage
\thispagestyle{empty}
\clearpage
\vspace*{\fill}
\begin{center}
    \begin{minipage}{\textwidth}
        \begin{center}
            \section*{Introducción}
            La siguiente memoria ilustra cómo acaban dos semanas de infierno terrenal. En ella, las numerosas entregas de prácticas y exámenes ---junto a otros muchos eventos de carácter extra-curricular--- no pudieron derrotar a los dos miembros de nuestro equipo ni a los demás compañeros (hasta donde saben nuestras fuentes). Este tutorial es la guinda final a una historia de desesperación, hundimiento emocional, altibajos pero, sobre todo, un intenso trabajo psicológico y de organización del cual estamos más que orgullosos.
        \end{center}
    \end{minipage}
\end{center}
\vfill

\newpage

\begin{center}
\section{Ejercicio 1}
\end{center}

\subsection*{\small ¿Qué se muestra en \texttt{Show results} de los dos nodos \texttt{TextViewer}? ¿Cuál es el porcentaje de instancias clasificadas correctamente?}

Respuesta here

\subsection*{\small ¿Cuál es la utilidad de crear flujos de conocimiento con esta interfaz de Weka?}

Respuesta here

\begin{center}
\section{Ejercicio 2}
\end{center}


\subsection*{\small De los conjuntos utilizados, ¿alguno parece más adecuado?}

Respuesta here

\subsection*{\small ¿Qué algoritmo de los analizados parece más adecuado? ¿Son los resultados del mejor algoritmo mucho mejores que los del resto?}

Respuesta here

\subsection*{\small Los resultados que se obtienen cambiando el criterio de evaluación, ¿guardan relación con los proporcionados en Percent correct? ¿Qué otro criterio se ha seleccionado? ¿Por qué?}

Respuesta here

\subsection*{\small Tras generar con el \emph{Explorer} el modelo que parece más adecuado, ¿es tan adecuado como parecía? ¿Por qué? Elije un modelo final justifiacndo tu respuesta.}

Respuesta here

\subsection*{\small ¿Por qué o para qué te parece adecuado el uso del \emph{Experiment}?}

Respuesta here

\newpage
\section{Problemas encontrados}

Respuesta here

\section{Conclusiones, opinión personal sobre el tutorial}

Respuesta here

\end{document}
