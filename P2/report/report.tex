\documentclass[12pt]{article}
\usepackage[spanish]{babel}
\usepackage{makeidx}
\usepackage[margin=1in]{geometry}  % set the margins to 1in on all sides
\usepackage{graphicx}              % to include figures
\usepackage{amsmath}               % great math stuff
\usepackage{amsfonts}              % for blackboard bold, etc
\usepackage{amsthm}                % better theorem environments
\usepackage{makeidx}               % index
\usepackage[utf8]{inputenc}        % now we have tildes!
\usepackage{wrapfig}               % images
\usepackage{listings}              % Unordered lists
\usepackage{hyperref}              % hyperlinks
\usepackage{xcolor}                % to colorize font
\usepackage{blindtext}             % to colorize font

\makeindex

\begin{document}

\begin{titlepage}

\newcommand{\HRule}{\rule{\linewidth}{0.5mm}} % Defines a new command for the horizontal lines, change thickness here

\center % Center everything on the page

%----------------------------------------------------------------------------------------
%	LOGO SECTION
%----------------------------------------------------------------------------------------

\textsc{\LARGE Universidad Carlos III de Madrid}\\[1.2cm] % Name of your university/college

%----------------------------------------------------------------------------------------
%	HEADING SECTIONS
%----------------------------------------------------------------------------------------

\includegraphics[width=9cm]{Logo}\\[1.2cm] % Include a department/university logo - this will require the graphicx package

\textsc{\Large Aprendizaje Automático}\\[0.5cm] % Major heading such as course name
\textsc{\large Grado en Ingeniería Informática}\\[0.6cm] % Minor heading such as course title
\textsc{\large Grupo 83}\\[0.5cm]

%----------------------------------------------------------------------------------------
%	TITLE SECTION
%----------------------------------------------------------------------------------------

\HRule \\[0.7cm]
{ \huge \bfseries Práctica 2: Aprendizaje basado en instancias}\\[0.4cm] % Title of your document
\HRule \\[0.7cm]

%----------------------------------------------------------------------------------------
%	AUTHOR SECTION
%----------------------------------------------------------------------------------------

\emph{Autores:}\\
Daniel \textsc{Medina García}\\ % Your name
Alejandro \textsc{Rodríguez Salamanca}\\[1.1cm] % Your name

%----------------------------------------------------------------------------------------
%	DATE SECTION
%----------------------------------------------------------------------------------------

{\large \today}\\ % Date, change the \today to a set date if you want to be precise

%----------------------------------------------------------------------------------------

\vfill % Fill the rest of the page with whitespace

\end{titlepage}

\tableofcontents

\newpage
\thispagestyle{empty}
\clearpage
\vspace*{\fill}
\begin{center}
    \begin{minipage}{\textwidth}
        \begin{center}
            \section*{Introducción}
            % TODO
        \end{center}
    \end{minipage}
\end{center}
\vfill

\newpage
\section{Recogida de información}

% TODO
% Descripción de las variables que representan el estado, así como su rango de valores.

\section{Clustering}

% TODO
% Descripción y justificación de los algoritmos utilizados para el proceso de clustering.
% Descripción de cualquier tratamiento sobre los datos que se lleve a cabo y de todos los pasos realizados.
% Descripción de las estructuras de datos utilizadas para el almacenamiento de la información generada en el proceso de clustering.
% Descripción de la función de pertenencia al cluster implementada.

\section{Generación del agente automático}

% TODO
% Descripción de la función de similitud entre instancias implementada.

\subsection{¿Por qué ha sido útil realizar clustering previa de las instancias?}

\subsection{¿Por qué es importante usar pocos atributos en técnicas de aprendizaje no supervisado?}

\subsection{¿Qué ventaja tiene el uso del aprendizaje basado en instancias con respecto al visto en la práctica 1?}

\subsection{¿Consideras que el agente funcionaría mejor si se introdujesen más ejemplos? ¿Por qué?}



\section{Evaluación de los agentes}

% TODO Será necesario evaluar el aprendizaje del agente automático de esta práctica. Para ello hay que realizar las siguientes tareas:

% Descripción y análisis de los resultados obtenidos en la fase de evaluación.

% 3. Evaluar para cada uno de los agentes, cómo evoluciona la distancia recorrida y enemigos muertos en cada instante de cada partida. Para ello realizar una gráfica o una tabla donde se muestre por un lado el tiempo vs distancia y otra con los fantasmas vs tiempo. Se recomienda hacer una media de todas las partidas jugadas, con lo que se realizarán dos gráficas por agente.

% 4. Una tabla resumen con las medias y desviaciones estándar de los agentes en los distintos mapas.

\newpage
\section{Conclusiones}

% TODO
% Conclusiones sobre la tarea realizada incluyendo apreciaciones m ́as generales como: para qu ́e puede ser u ́til el modelo obtenido, si al realizar la pr ́actica se os han ocurrido otros dominios en que se pueda aplicar aprendizaje autom ́atico, etc.

\vspace{0.2cm}

\centerline{\textbf{Problemas encontrados}}

\vspace{0.5cm}

% TODO

\vspace{0.5cm}

\centerline{\textbf{Comentarios personales}}

\vspace{0.5cm}

% TODO

\end{document}
