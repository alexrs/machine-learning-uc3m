\documentclass[12pt]{article}
\usepackage{makeidx}
\usepackage[margin=1in]{geometry}  % set the margins to 1in on all sides
\usepackage{graphicx}              % to include figures
\usepackage{amsmath}               % great math stuff
\usepackage{amsfonts}              % for blackboard bold, etc
\usepackage{amsthm}                % better theorem environments
\usepackage{makeidx}               % index
\usepackage[utf8]{inputenc}        % now we have tildes!
\usepackage{wrapfig}               % images
\usepackage{listings}              % Unordered lists

% various theorems, numbered by section

\makeindex



\newtheorem{thm}{Theorem}[section]
\newtheorem{lem}[thm]{Lemma}
\newtheorem{prop}[thm]{Proposition}
\newtheorem{cor}[thm]{Corollary}
\newtheorem{conj}[thm]{Conjecture}

\graphicspath{{maps/}}

\lstset{numberstyle=\scriptsize\ttfamily, numbersep=7pt, captionpos=b}
\lstset{basicstyle=\small\ttfamily}
\lstset{framesep=2pt}

\DeclareMathOperator{\id}{id}

\newcommand{\bd}[1]{\mathbf{#1}}  % for bolding symbols
\newcommand{\RR}{\mathbb{R}}      % for Real numbers
\newcommand{\ZZ}{\mathbb{Z}}      % for Integers
\newcommand{\col}[1]{\left[\begin{matrix} #1 \end{matrix} \right]}
\newcommand{\comb}[2]{\binom{#1^2 + #2^2}{#1+#2}}



\begin{document}

\begin{titlepage}

\newcommand{\HRule}{\rule{\linewidth}{0.5mm}} % Defines a new command for the horizontal lines, change thickness here

\center % Center everything on the page

%----------------------------------------------------------------------------------------
%	HEADING SECTIONS
%----------------------------------------------------------------------------------------

\textsc{\LARGE Universidad Carlos III de Madrid}\\[1.5cm] % Name of your university/college
\textsc{\Large Aprendizaje Automático}\\[0.5cm] % Major heading such as course name
\textsc{\large Computer Science Engineering}\\[0.5cm] % Minor heading such as course title

%----------------------------------------------------------------------------------------
%	TITLE SECTION
%----------------------------------------------------------------------------------------

\HRule \\[0.4cm]
{ \huge \bfseries Tutorial 1: Plataforma PacMan}\\[0.4cm] % Title of your document
\HRule \\[1.5cm]

%----------------------------------------------------------------------------------------
%	AUTHOR SECTION
%----------------------------------------------------------------------------------------


% If you don't want a supervisor, uncomment the two lines below and remove the section above
\emph{Authors:}\\
Daniel \textsc{Medina García}\\ % Your name
Alejandro \textsc{Rodríguez Salamanca}\\[3cm] % Your name

%----------------------------------------------------------------------------------------
%	DATE SECTION
%----------------------------------------------------------------------------------------

{\large \today}\\[3cm] % Date, change the \today to a set date if you want to be precise

%----------------------------------------------------------------------------------------
%	LOGO SECTION
%----------------------------------------------------------------------------------------

%\includegraphics{Logo}\\[1cm] % Include a department/university logo - this will require the graphicx package

%----------------------------------------------------------------------------------------

\vfill % Fill the rest of the page with whitespace

\end{titlepage}

\tableofcontents

\newpage

\begin{center}
\section{Los ficheros de datos}
\end{center}

\subsection{¿Cuántos atributos de entrada tiene el fichero de datos?
¿De qué tipo son?}

Uno. Decir tipo y demás y por qué el de clase no es de entrada.

\subsection{¿Podría un algoritmo de aprendizaje automático identificar esa
función con los datos que hay en ese fichero? ¿Por qué?}

No porque mi respuesta es de libro.

\newpage

\begin{center}
\section{Clasificar con ZeroR}
\end{center}

\subsection{¿Qué resultado en términos de instancias correctas ofrece el algoritmo
ZeroR?}

Respuesta

\subsection{¿Qué ocurre si se selecciona otro algoritmo de clasificación permitido
para ese conjunto de datos?}

Respuesta

\subsection{¿Cuáles son las diferencias al repetir los pasos anteriores con el otro
fichero \texttt{badges\_plain.arff}?}

Respuesta

\subsection{¿Qué ocurre si seleccionamos el algoritmo trees / ID3 en el segundo
fichero?}

Respuesta

\newpage

\begin{center}
\section{Generando nuevos atributos}
\end{center}

\subsection{Propón 6 nuevos atributos y explica por qué los has elegido.}

Respuesta

\subsection{¿Cuántos atributos tiene el fichero \texttt{badges1.arff} y de qué tipo
son?}

Respuesta

\subsection{¿Qué otro tipo de información estadística se muestra sobre los
atributos? Tras pulsar el botón ``Visualize all'' indica qué se muestra y si
hay algún atributo que no se visualice.}

Respuesta

\subsection{Genera un clasificador con ZeroR, ¿qué ocurre? Compara los resultados con
los obtenidos en el ejercicio anterior.}

Respuesta

\subsection{Genera un clasificador con trees / ID3, ¿qué ocurre? ¿Qué se podría hacer
para solucionar este problema?}

No te deja ejecutarlo, por qué? Creo que tenía que ver con el tipo de datos

\newpage

\begin{center}
\section{Clasificar con ID3: resolviendo problemas}
\end{center}

\subsection{¿Qué información aparece en el desplegable tras abrir la pestaña
\emph{Classify / ZeroR / Capabilities}?}

Respuesta

\subsection{¿Qué información proporciona ``more''?}

Respuesta

\paragraph{Los atributos de entrada pueden modificarse a través de tareas de
preprocesamiento. En los siguientes pasos vamos a modificar ciertos atributos
de \texttt{badges1.arff} para que pueda clasificarse con ID3.}

\subsection{¿Qué efecto tiene el filtro \emph{Filter / unsupervised / attribute
/ Discretize} sobre el conjunto de datos con \emph{bins} igual a 5?}

Respuesta

\subsection{¿Cuántas instancias clasifica bien cuando marcamos \emph{Use
training set} en \emph{Test options}? ¿Qué porcentaje representa?}

Respuesta

\subsection{¿Qué crees que indica la ``matriz de confusión''?}

Respuesta

\subsection{¿Cuántas instancias de cada tipo se han clasificado mal?}

Respuesta

\paragraph{Ahora pulsamos \emph{More Options} y seleccionamos la opción
\emph{Outuput predictions}}
\subsection{¿Cuál es la primera instancia del conjunto de entrenamiento que se
clasifica mal? ¿Por qué?}

Respuesta

\subsection{¿Cómo se clasificaría la instancia ``Eloisa Figueroa''? ¿Cuáles son
los atributos de este nombre? ¿Qué ocurre con los valores de esta instancia
si utilizas el filtro usado anteriormente?}

Respuesta

\paragraph{A continuación modificamos el fichero original introduciendo el nombre
anterior y cambiamos la clase a ``positiva'', teniendo en cuenta que si contiene
enumerados y se introduce un nuevo valor hay que especificarlo también en la
definición de los valores posibles del enumerado. Después volvemos a generar el
clasificador con ZeroR y training set seleccionado.}
\subsection{¿Cómo se clasifica la instancia nueva?}

Respuesta

\newpage

\begin{center}
\section{Clasificar con J48 (C4.5)}
\end{center}

\paragraph{Volvemos a la pestaña de preproceso para cargar \texttt{badges1.arff}
, usando el training set para volver a generar el clasificador.}

\subsection{Pregunta}

Respuesta

\newpage

\begin{center}
\section{Pregunta 1}
\end{center}

\subsection{Pregunta}

Respuesta

\newpage

\begin{center}
\section{Pregunta 1}
\end{center}

\subsection{Pregunta}

Respuesta

\end{document}
