% !TEX encoding = UTF-8 Unicode
\documentclass[12pt]{article}
\usepackage{makeidx}
\usepackage[margin=1in]{geometry}  % set the margins to 1in on all sides
\usepackage{graphicx}              % to include figures
\usepackage{amsmath}               % great math stuff
\usepackage{amsfonts}              % for blackboard bold, etc
\usepackage{amsthm}                % better theorem environments
\usepackage{makeidx}               % index
\usepackage[utf8]{inputenc}        % now we have tildes!
\usepackage{wrapfig}               % images
\usepackage{listings}              % Unordered lists

\makeindex

\begin{document}

\begin{titlepage}

\newcommand{\HRule}{\rule{\linewidth}{0.5mm}} % Defines a new command for the horizontal lines, change thickness here

\center % Center everything on the page

%----------------------------------------------------------------------------------------
%	HEADING SECTIONS
%----------------------------------------------------------------------------------------

\textsc{\LARGE Universidad Carlos III de Madrid}\\[1.5cm] % Name of your university/college
\textsc{\Large Aprendizaje Automático}\\[0.5cm] % Major heading such as course name
\textsc{\large Computer Science Engineering}\\[0.5cm] % Minor heading such as course title

%----------------------------------------------------------------------------------------
%	TITLE SECTION
%----------------------------------------------------------------------------------------

\HRule \\[0.4cm]
{ \huge \bfseries Tutorial 2: Introducción a Weka}\\[0.4cm] % Title of your document
\HRule \\[1.5cm]

%----------------------------------------------------------------------------------------
%	AUTHOR SECTION
%----------------------------------------------------------------------------------------

\emph{Authors:}\\
Daniel \textsc{Medina Garcí­a}\\ % Your name
Alejandro \textsc{Rodrí­guez Salamanca}\\[3cm] % Your name

%----------------------------------------------------------------------------------------
%	DATE SECTION
%----------------------------------------------------------------------------------------

{\large \today}\\[3cm] % Date, change the \today to a set date if you want to be precise

%----------------------------------------------------------------------------------------
%	LOGO SECTION
%----------------------------------------------------------------------------------------

%\includegraphics{Logo}\\[1cm] % Include a department/university logo - this will require the graphicx package

%----------------------------------------------------------------------------------------

\vfill % Fill the rest of the page with whitespace

\end{titlepage}

\tableofcontents

\newpage

\begin{center}
\section{Los ficheros de datos}
\end{center}

\subsection{\small ¿Cuántos atributos de entrada tiene el fichero de datos?
¿De qué tipo son?}

\begin{description}
  \item[badges.arff] Este archivo contiene un atributo de entrada de tipo String que se corresponde con un nombre, y un atributo de salida, la clase a la que pertenece.
  \item[badges plain.arff] El contenido de este archivo es el mismo que el de  badges.arff con la diferencia de que los nombres están previamente enumerados.\ldots
\end{description}

\subsection{\small ¿Podrí­a un algoritmo de aprendizaje automático identificar esa
función con los datos que hay en ese fichero? ¿Por qué?}

No porque mi respuesta es de libro.

\newpage

\begin{center}
\section{Clasificar con ZeroR}
\end{center}

\subsection{\small ¿Qué resultado en términos de instancias correctas ofrece el algoritmo
ZeroR?}

ZeroR indica un resultado del 51.0204\% de instancias correctas.

\subsection{\small ¿Qué ocurre si se selecciona otro algoritmo de clasificación permitido
para ese conjunto de datos?}

El algoritmo elegido ha sido meta/MultiScheme. El resultado de instancias correctamente clasificadas ha sido el mismo que el obtenido con ZeroR, 51.0204\%.

\subsection{\small ¿Cuáles son las diferencias al repetir los pasos anteriores con el otro
fichero \texttt{badges\_plain.arff}?}

Al ser en este caso el conjunto de datos finito, ya que está enumerado, podemos usar un mayor número de algoritmos. Volviendo a probar con ZeroR, podemos observar que continuamos obteniendo el mismo resultado. Esto es debido a que ZeroR toma como resultado la clase a la que pertenezcan la mayoría de los datos.

\subsection{\small ¿Qué ocurre si seleccionamos el algoritmo trees / ID3 en el segundo
fichero?}

Si probamos con ID3 podemos observar que el porcentaje de instancias correctamente clasificadas es del 100\%. (Esto indica que existe un sobreajuste y es probable, como hemos discutido en el ejercicio 1, que ID3 no haya aprendido nada, y simplemente haya memorizado).

\newpage

\begin{center}
\section{Generando nuevos atributos}
\end{center}

\subsection{\small Propón 6 nuevos atributos y explica por qué los has elegido.}

Número de letras del nombre. Número de vocales. Número de consonantes. Número de espacios. Inicial del nombre. Inicial del apellido.

\subsection{\small ¿Cuántos atributos tiene el fichero \texttt{badges1.arff} y de qué tipo
son?}

8

\subsection{\small ¿Qué otro tipo de información estadí­stica se muestra sobre los
atributos? Tras pulsar el botón ``Visualize all'' indica qué se muestra y si
hay algún atributo que no se visualice.}

Respuesta

\subsection{\small Genera un clasificador con ZeroR, ¿qué ocurre? Compara los resultados con
los obtenidos en el ejercicio anterior.}

Respuesta

\subsection{\small Genera un clasificador con trees / ID3, ¿qué ocurre? ¿Qué se podrí­a hacer
para solucionar este problema?}

No te deja ejecutarlo, por qué? Creo que tení­a que ver con el tipo de datos

\newpage

\begin{center}
\section{Clasificar con ID3: resolviendo problemas}
\end{center}

\subsection{\small ¿Qué información aparece en el desplegable tras abrir la pestaí±a
\emph{Capabilities}? ¿Qué información proporciona ``more''?}

Respuesta

\paragraph{\small Los atributos de entrada pueden modificarse a través de tareas de
preprocesamiento. En los siguientes pasos vamos a modificar ciertos atributos
de \texttt{badges1.arff} para que pueda clasificarse con ID3.}

\subsection{\small ¿Qué efecto tiene el filtro de discretización sobre el conjunto de
datos con \emph{bins} igual a 5?}

Respuesta

\subsection{\small ¿Cuántas instancias clasifica bien cuando marcamos \emph{Use
training set}? ¿Qué porcentaje representa? ¿Qué crees que indica la ``matriz de
confusión''? ¿Cuántas instancias de cada tipo se han clasificado mal?}

Respuesta

\paragraph{\small Ahora se selecciona la opción \emph{Outuput predictions}}
\subsection{\small ¿Cuál es la primera instancia del conjunto de entrenamiento que se
clasifica mal? ¿Por qué?}

Respuesta

\subsection{\small ¿Cómo se clasificarí­a la instancia ``Eloisa Figueroa''? ¿Cuáles son
los atributos de este nombre? ¿Qué ocurre con los valores de esta instancia
si utilizas el filtro usado anteriormente?}

Respuesta

\paragraph{\small A continuación modificamos el fichero original introduciendo el nombre
anterior y cambiamos la clase a ``positiva'', teniendo en cuenta que si contiene
enumerados y se introduce un nuevo valor hay que especificarlo también en la
definición de los valores posibles del enumerado. Después volvemos a generar el
clasificador con ZeroR y training set seleccionado.}
\subsection{\small ¿Cómo se clasifica la instancia nueva?}

Respuesta

\newpage

\begin{center}
\section{Clasificar con J48 (C4.5)}
\end{center}

\paragraph{\small Volvemos a la pestaí±a de preproceso para cargar \texttt{badges1.arff}
y volver a generar el clasificador usando la opción de \emph{training set.}}

\subsection{\small ¿Cuántas hojas tiene el árbol generado con J48?}

Respuesta

\subsection{\small ¿Cuántas instancias del conjunto de entrenamiento clasifica bien?
¿Qué porcentaje representa? ¿Cuántas instancias de cada tipo se han clasificado
mal? ¿Cómo se clasificarí­a la instancia ``Eloisa Figueroa''?}

Respuesta

\subsection{\small ¿Elegirí­as este modelo o el generado por ID3? ¿Por qué?}

Respuesta

\subsection{\small ¿Hemos encontrado la función exacta para generar las etiquetas?
¿Por qué lo sabes?}

Respuesta

\newpage

\section{Utilizando más atributos con J48 (C4.5)}

\paragraph{\small Es momento de volver a la pestaí±a de preproceso y generar un
nuevo atributo que calcule el número de vocales. Después se grabará el conjunto
de datos como \texttt{badges1-2.arff}, y con él se construirá un clasificador
con J48. Una vez generado, se anotan el porcentaje de instancias bien
clasificadas y la matriz de confusión, tras lo cual visualizaremos el árbol
generado.}

\subsection{\small ¿Qué indican los números que aparecen en las hojas del árbol?}

Respuesta

\subsection{\small ¿Qué efecto tiene aumentar el valor de ``Jitter'' en la gráfica que
relaciona el nuevo atributo con la clase?}

Respuesta

\subsection{\small ¿Podrí­as decir cuál es el rango de vocales más común en el fichero
proporcionado? ¿Se te ocurre algún otro atributo relacionado que pueda aportar
información?}

Respuesta

\subsection{\small Tras todos estos resultados, ¿qué caracterí­sticas o cualidades
crees que deben tener los atributos para maximizar el éxito de los algoritmos
de aprendizaje automático?}

Respuesta

\newpage

\section{Balanceado de datos, selección de caracterí­sticas y otros filtros}

\paragraph{\small Para este apartado se ha de cargar en Weka \texttt{adult-data.arff}.}

\subsection{\small ¿Cuántos atributos de entrada tiene este fichero? ¿Cuántas
instancias de entrenamiento?}

Respuesta

\paragraph{\small Ahora se ejecuta el clasificador ZeroR con \emph{cross-validation}.}

\subsection{\small ¿Qué resultados aparecen? Explica el resultado.}

Respuesta

\paragraph{\small A continuación se evalua el clasificador solamente con las
instancias que figuren en el fichero \texttt{adult-test.arff}.}

\subsection{\small ¿Qué resultados aparecen? ¿Son estos resultados comparables a los
anteriores? ¿Por qué?}

Respuesta

\paragraph{\small Se repite este procedimiento con el J48 (en lugar de ZeroR) usando
\emph{cross-validation} y \emph{Supplied test set}.}

\subsection{\small ¿Qué resultados aparecen? ¿Qué porcentaje de mejora ha obtenido
respecto a los resultados del ZeroR?}

Respuesta

\subsection{\small ¿Qué proporción de datos hay de cada clase? ¿Crees que este
porcentaje es apropiado para que un algoritmo de aprendizaje automático aprenda
bien?}

Respuesta

\paragraph{\small Se procede a modificar las instancias de entrenamiento para que
tengan un porcentaje similar entre las dos clases.}
\subsection{\small ¿Qué ocurre con el atributo de salida? ¿Ha descendido el número de
ejemplos de entrenamiento?}

Respuesta

\paragraph{\small Tras aplicar este filtro, se evalúa de nuevo con \emph{cross-validation}
 y supplied test set los algoritmos ZeroR y J48.}

\subsection{\small ¿Qué resultados dan los algoritmos? ¿Qué resultado crees que es
mejor? ¿Por qué?}

Respuesta

\paragraph{\small Se eliminan uno o dos atributos que no se creen útiles para el
algoritmo de aprendizaje.}
\subsection{\small ¿Cuáles has eliminado? ¿Por qué? ¿Qué es lo que ocurre al repetir la evaluación anterior?}

Respuesta

\paragraph{\small Por último se aplica el filtro de normalización para los atributos
numéricos.}

\subsection{\small ¿Qué resultados se obtienen?}

Respuesta

\subsection{\small Después del procesamiento de datos que has realizado en este
apartado, ¿crees que esto ayuda al proceso de aprendizaje? ¿Por qué? ¿Cuál es
el mejor resultado obtenido? Justifí­calo.}

Respuesta

\end{document}
