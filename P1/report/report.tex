\documentclass[12pt]{article}
\usepackage{makeidx}
\usepackage[margin=1in]{geometry}  % set the margins to 1in on all sides
\usepackage{graphicx}              % to include figures
\usepackage{amsmath}               % great math stuff
\usepackage{amsfonts}              % for blackboard bold, etc
\usepackage{amsthm}                % better theorem environments
\usepackage{makeidx}               % index
\usepackage[utf8]{inputenc}        % now we have tildes!
\usepackage{wrapfig}               % images
\usepackage{listings}              % Unordered lists

% various theorems, numbered by section

\makeindex



\newtheorem{thm}{Theorem}[section]
\newtheorem{lem}[thm]{Lemma}
\newtheorem{prop}[thm]{Proposition}
\newtheorem{cor}[thm]{Corollary}
\newtheorem{conj}[thm]{Conjecture}

\graphicspath{{maps/}}

\lstset{numberstyle=\scriptsize\ttfamily, numbersep=7pt, captionpos=b}
\lstset{basicstyle=\small\ttfamily}
\lstset{framesep=2pt}

\DeclareMathOperator{\id}{id}

\newcommand{\bd}[1]{\mathbf{#1}}  % for bolding symbols
\newcommand{\RR}{\mathbb{R}}      % for Real numbers
\newcommand{\ZZ}{\mathbb{Z}}      % for Integers
\newcommand{\col}[1]{\left[\begin{matrix} #1 \end{matrix} \right]}
\newcommand{\comb}[2]{\binom{#1^2 + #2^2}{#1+#2}}

\makeatletter
\renewcommand\thesection{}
\renewcommand\thesubsection{\@arabic\c@section.\@arabic\c@subsection}
\makeatother

\begin{document}

\begin{titlepage}

\newcommand{\HRule}{\rule{\linewidth}{0.5mm}} % Defines a new command for the horizontal lines, change thickness here

\center % Center everything on the page

%----------------------------------------------------------------------------------------
%	HEADING SECTIONS
%----------------------------------------------------------------------------------------

\textsc{\LARGE Universidad Carlos III de Madrid}\\[1.5cm] % Name of your university/college
\textsc{\Large Aprendizaje Automático}\\[0.5cm] % Major heading such as course name
\textsc{\large Computer Science Engineering}\\[0.5cm] % Minor heading such as course title

%----------------------------------------------------------------------------------------
%	TITLE SECTION
%----------------------------------------------------------------------------------------

\HRule \\[0.4cm]
{ \huge \bfseries Práctica 1: Clasificación y Predicción}\\[0.4cm] % Title of your document
\HRule \\[1.5cm]

%----------------------------------------------------------------------------------------
%	AUTHOR SECTION
%----------------------------------------------------------------------------------------


% If you don't want a supervisor, uncomment the two lines below and remove the section above
\emph{Authors:}\\
Daniel \textsc{Medina García}\\ % Your name
Alejandro \textsc{Rodríguez Salamanca}\\[3cm] % Your name

%----------------------------------------------------------------------------------------
%	DATE SECTION
%----------------------------------------------------------------------------------------

{\large \today}\\[3cm] % Date, change the \today to a set date if you want to be precise

%----------------------------------------------------------------------------------------
%	LOGO SECTION
%----------------------------------------------------------------------------------------

%\includegraphics{Logo}\\[1cm] % Include a department/university logo - this will require the graphicx package

%----------------------------------------------------------------------------------------

\vfill % Fill the rest of the page with whitespace

\end{titlepage}

\tableofcontents

\newpage

\section*{Introducción}

\huge Abstract \small

\section{Fase 1}

Explicación de la función de extracción de características que se ha programado, así como del mecanismo para incluir datos futuros en una misma instancia.

\section{Fase 2}

Explicación de la experimentación tal como se explica en el enunciado. Debe incluir la justificación de los algoritmos seleccionados, de los atributos seleccionados y de cualquier tratamiento sobre los datos que se haya llevado a cabo. Se concluirá con un análisis de los resultados producidos por los algoritmos elegidos y justificación de la elección del modelo final.

\section{Fase 2}

Igual que en la fase 2, pero con cada modelo de regresión.

\section{Preguntas}

\emph{¿Qué diferencias hay a la hora de aprender esos modelos con instancias provenientes de un agente controlado por un humano y uno automático?}

\emph{¿Crees que los resultados del modelo de regresión a 5 turnos vista guardan relación con los de 2 turnos? ¿Por qué?}

\emph{Si quisieras transformar la tarea de regresión en clasificación ¿Qué tendrías que hacer? ¿Cuál crees que podría ser la aplicación práctica de predecir la puntuación?}

\emph{¿Qué ventajas puede aportar predecir la puntuación respecto a la clasificación de la acción? Justifica tu respuesta.}

\emph{¿Crees que se podría conseguir alguna mejora en la clasificación incorporando un atributo que indicase si la puntuación en el instante actual ha descendido o ha bajado?}

\section{Conclusiones}

Conclusiones técnicas sobre la tarea que se ha realizado.
\begin{itemize}
  \item Apreciaciones más generales como: para qué puede ser útil el modelo obtenido, si al realizar la práctica se os han ocurrido otros dominios en que se pueda aplicar aprendizaje automático, etc.
  \item Descripción de los problemas encontrados a la hora de realizar esta práctica.
  \item Comentarios personales. Opinión acerca de la práctica. Dificultades encontradas, críticas, etc.
\end{itemize}

\end{document}
